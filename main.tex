\documentclass[conference]{IEEEtran}
\IEEEoverridecommandlockouts
% The preceding line is only needed to identify funding in the first footnote. If that is unneeded, please comment it out.
\usepackage{cite}
\usepackage{amsmath,amssymb,amsfonts}
\usepackage{algorithmic}
\usepackage{graphicx}
\usepackage{textcomp}
\usepackage{xcolor}
\def\BibTeX{{\rm B\kern-.05em{\sc i\kern-.025em b}\kern-.08em
    T\kern-.1667em\lower.7ex\hbox{E}\kern-.125emX}}
\begin{document}

\title{Addressing Crash Recovery with Persistent Memory Allocation\\
{\footnotesize \textsuperscript{}}
}

\author{\IEEEauthorblockN{\textsuperscript{st} Nicholas Stone}
\IEEEauthorblockA{\textit{Xavier University} \\
Cincinnati Ohio, USA \\
Stonen2@Xavier.edu}
\and
\IEEEauthorblockN{\textsuperscript{nd} Michael Spear}
\IEEEauthorblockA{\textit{Lehigh University)} \\
\textit{Bethlehem Pennsylvania, USA}\\
spear@lehigh.edu}
\and
\IEEEauthorblockN{\textsuperscript{rd} Roberto Palmieri}
\IEEEauthorblockA{\textit{Lehigh University} \\
\textit{Bethlehem Pennsylvania, USA}\\
palmieri@lehigh.edu}
}
\maketitle

\begin{abstract}
Persistent memory is a combination of both volatile memory and long-term storage.  This means Persistent memory is able to maintain the speed of volatile memory such as Dynamic Random-Access Memory (DRAM) and is like the long-term storage of devices such as Solid-State Drive (SSD) and Hard Disk Drive (HDD), in its ability to maintain its state without consuming power. Furthermore, since persistent memory has Byte addressability and long-term storage, allocators can be used to maintain Data-structures that will persist even after power failure or program crashes. While persistent memory has unique characteristics that DRAM does not have, persistent memory has a few draw backs. Excessive writes to a single memory cell can destroy it and writes on persistent memory are much slower than reads. In this project we will be creating a memory allocator that will leverage both DRAM and Non-Volatile Random-Access Memory (NVRAM) to perform smart memory allocation that will be able to allow a data structure’s integrity to be preserved despite crashes and power failures without excessive writing to the persistent memory.
\end{abstract}

\begin{IEEEkeywords}
HDD,SSD,allocator,NVRAM,Persistent Memory
\end{IEEEkeywords}

\section{Introduction}
This document is a model and instructions for \LaTeX.
Please observe the conference page limits. 

\section{Background}

\subsection{Persistent Memory Non-Volatile Random Access Memory}
Place Hold

\subsection{Memory Allocation}
Place Hold





\section{Problems}
Place Holder

\section{Solutions}
Place Holder

\section{evaluation}
Place Holder

\section{Conclusions and Future Work}
Place Holder

\section*{Acknowledgment}

The preferred spelling of the word ``acknowledgment'' in America is without 
an ``e'' after the ``g''. Avoid the stilted expression ``one of us (R. B. 
G.) thanks $\ldots$''. Instead, try ``R. B. G. thanks$\ldots$''. Put sponsor 
acknowledgments in the unnumbered footnote on the first page.

\section*{References}

Please number citations consecutively within brackets \cite{IEEEhowto:IEEEtranpage}. The 
sentence punctuation follows the bracket \cite{b2}. Refer simply to the reference 
number, as in \cite{b3}---do not use ``Ref. \cite{b3}'' or ``reference \cite{b3}'' except at 
the beginning of a sentence: ``Reference \cite{b3} was the first $\ldots$''

Number footnotes separately in superscripts. Place the actual footnote at 
the bottom of the column in which it was cited. Do not put footnotes in the 
abstract or reference list. Use letters for table footnotes.

Unless there are six authors or more give all authors' names; do not use 
``et al.''. Papers that have not been published, even if they have been 
submitted for publication, should be cited as ``unpublished'' \cite{b4}. Papers 
that have been accepted for publication should be cited as ``in press'' \cite{b5}. 
Capitalize only the first word in a paper title, except for proper nouns and 
element symbols.

For papers published in translation journals, please give the English 
citation first, followed by the original foreign-language citation \cite{b6}.

\bibliographystyle{./bibliography/IEEEtran}
\bibliography{./bibliography/IEEEabrv,./bibliography/IEEEexample}

\vspace{12pt}
\color{red}
IEEE conference templates contain guidance text for composing and formatting conference papers. Please ensure that all template text is removed from your conference paper prior to submission to the conference. Failure to remove the template text from your paper may result in your paper not being published.

https://people.cs.umass.edu/~emery/pubs/berger-oopsla2002.pdf

https://people.cs.umass.edu/~emery/pubs/berger-oopsla2002.pdf

http://www.vldb.org/pvldb/vol8/p497-chatzistergiou.pdf

https://link.springer.com/chapter/10.1007/978-3-662-53426-7_23

https://dl.acm.org/citation.cfm?id=3054780

https://dl.acm.org/citation.cfm?id=2984019




\end{document}

