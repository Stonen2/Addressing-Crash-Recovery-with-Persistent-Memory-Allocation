\documentclass[conference]{IEEEtran}
\IEEEoverridecommandlockouts
% The preceding line is only needed to identify funding in the first footnote. If that is unneeded, please comment it out.
\usepackage{cite}
\usepackage{amsmath,amssymb,amsfonts}
\usepackage{algorithmic}
\usepackage{graphicx}
\usepackage{textcomp}
\usepackage{xcolor}
\def\BibTeX{{\rm B\kern-.05em{\sc i\kern-.025em b}\kern-.08em
    T\kern-.1667em\lower.7ex\hbox{E}\kern-.125emX}}
\begin{document}

\title{Addressing Crash Recovery with Persistent Memory Allocation\\
{\footnotesize \textsuperscript{}}
}

\author{\IEEEauthorblockN{\textsuperscript{st} Nicholas Stone}
\IEEEauthorblockA{\textit{Xavier University} \\
Cincinnati Ohio, USA \\
Stonen2@Xavier.edu}
\and
\IEEEauthorblockN{\textsuperscript{nd} Michael Spear}
\IEEEauthorblockA{\textit{Lehigh University)} \\
\textit{Bethlehem Pennsylvania, USA}\\
spear@lehigh.edu}
\and
\IEEEauthorblockN{\textsuperscript{rd} Roberto Palmieri}
\IEEEauthorblockA{\textit{Lehigh University} \\
\textit{Bethlehem Pennsylvania, USA}\\
palmieri@lehigh.edu}
}
\maketitle

\begin{abstract}
Persistent memory is a combination of both volatile memory and long-term storage.  This means Persistent memory is able to maintain the speed of volatile memory such as Dynamic Random-Access Memory (DRAM) and is like the long-term storage of devices such as Solid-State Drive (SSD) and Hard Disk Drive (HDD), in its ability to maintain its state without consuming power. Furthermore, since persistent memory has Byte addressability and long-term storage, allocators can be used to maintain Data-structures that will persist even after power failure or program crashes. While persistent memory has unique characteristics that DRAM does not have, persistent memory has a few draw backs. Excessive writes to a single memory cell can destroy it and writes on persistent memory are much slower than reads. In this project we will be creating a memory allocator that will leverage both DRAM and Non-Volatile Random-Access Memory (NVRAM) to perform smart memory allocation that will be able to allow a data structure’s integrity to be preserved despite crashes and power failures without excessive writing to the persistent memory.
\end{abstract}

\begin{IEEEkeywords}
HDD,SSD,allocator,NVRAM,Persistent Memory
\end{IEEEkeywords}

\section{Introduction}
Non-Volatile Random-Access Memory (NVRAM) is a new kind of persistent memory that has a few unique characteristics that are different than that of Random-Access Memory (RAM).  The first is that this new form of technology allows for byte addressable long-term storage means that operate and execute several orders of magnitude faster than that of both Hard Disk Drives (HDD) and Solid-State Drives (SSD). Another unique factor of NVRAM is that this form of technology has a limited number of writes before the memory cells becomes unusable. If a program is not implemented properly or poorly the program can destroy memory cells rapidly thus damaging the technology and rendering it more expensive to use than typical RAM. One of the underlying tools that a programmer does not need to worry about usually is the allocator that a program is running or using.  Meaning that if a programmer was to write a program that was constantly writing to the same memory cells this could lead to a short life span of the NVRAM technology. It then becomes critical the smart memory allocations and tactical memory writes is used to leverage the NVRAM technology to be most effective. 
Memory allocators have been used in systems since the early 1960s with the buddy memory allocation being implemented and used in 1963. As described above NVRAM has some unique properties that will enhance the runtime and performance of programs. However, if the same memory allocators that are used for RAM are used for NVRAM a large performance increase might not be experienced since these tactics are designed to run efficiently on RAM. The Memory allocating tactics that are used in this paper and program inherit some of the same or similar tactics to those that run on RAM but have a few unique characteristics that make them different. The first is that while locality is important for RAM technologies locality is critical for NVRAM technologies. Since NVRAM allows for data to persist even when a power cycle happens this means that any data in memory will be preserved on start up. Thus, allowing for a second program or an allocator to go through and find and restore all data that a program would have been using before a power cycle.  If a program can execute a systematic approach to memory allocation, then a simple algorithm can be implemented to then restore data in a quick and efficient manner. The second thing is that custom memory allocators may become more beneficial on NVRAM technologies than that of RAM. This is due to the fact that NVRAM could be leveraged in a variety of ways by many different systems, meaning that a DMBS might use NVRAM to hold ‘Hot’ Tuples while a Cloud system might use allocation tactics differently.  The following memory allocator aims to see what kind of allocation tactics yields the best performance for programs. 


\section{Background}

\subsection{Persistent Memory Non-Volatile Random Access Memory}
Place Hold

\subsection{Memory Allocation}
Place Hold





\section{Problems}
Place Holder

\section{Solutions}
Place Holder

\section{evaluation}
Place Holder

\section{Conclusions and Future Work}
Place Holder

\section*{Acknowledgment}

The preferred spelling of the word ``acknowledgment'' in America is without 
an ``e'' after the ``g''. Avoid the stilted expression ``one of us (R. B. 
G.) thanks $\ldots$''. Instead, try ``R. B. G. thanks$\ldots$''. Put sponsor 
acknowledgments in the unnumbered footnote on the first page.

\section*{References}

Please number citations consecutively within brackets \cite{IEEEhowto:IEEEtranpage}. The 
sentence punctuation follows the bracket \cite{b2}. Refer simply to the reference 
number, as in \cite{b3}---do not use ``Ref. \cite{b3}'' or ``reference \cite{b3}'' except at 
the beginning of a sentence: ``Reference \cite{b3} was the first $\ldots$''

Number footnotes separately in superscripts. Place the actual footnote at 
the bottom of the column in which it was cited. Do not put footnotes in the 
abstract or reference list. Use letters for table footnotes.

Unless there are six authors or more give all authors' names; do not use 
``et al.''. Papers that have not been published, even if they have been 
submitted for publication, should be cited as ``unpublished'' \cite{b4}. Papers 
that have been accepted for publication should be cited as ``in press'' \cite{b5}. 
Capitalize only the first word in a paper title, except for proper nouns and 
element symbols.

For papers published in translation journals, please give the English 
citation first, followed by the original foreign-language citation \cite{b6}.

\bibliographystyle{./bibliography/IEEEtran}
\bibliography{./bibliography/IEEEabrv,./bibliography/IEEEexample}

\vspace{12pt}
\color{red}
IEEE conference templates contain guidance text for composing and formatting conference papers. Please ensure that all template text is removed from your conference paper prior to submission to the conference. Failure to remove the template text from your paper may result in your paper not being published.

https://people.cs.umass.edu/~emery/pubs/berger-oopsla2002.pdf

https://people.cs.umass.edu/~emery/pubs/berger-oopsla2002.pdf

http://www.vldb.org/pvldb/vol8/p497-chatzistergiou.pdf

https://link.springer.com/chapter/10.1007/978-3-662-53426-7_23

https://dl.acm.org/citation.cfm?id=3054780

https://dl.acm.org/citation.cfm?id=2984019




\end{document}

